\documentclass[14pt, gost]{termpaper}

\usepackage{lipsum}

\begin{document}
\maketitle

\pagebreak
\newfontfamily\myfont{CMU Sans Serif}
\begin{center}
    \hspace{0pt}
    \vfill
    {\Huge\myfont
        Текст ВКР или учебной практики пишется не ради зачета, а чтобы люди его прочитали, поняли как круто Вы все сделали, и могли продолжить с того места, где Вы остановились.
        \vspace{2em}

        Повторять эту страницу в тексте вашей работы \emph{нельзя}.}
    \vfill
    \hspace{0pt}
\end{center}
\pagebreak

\setcounter{tocdepth}{2}
\tableofcontents

\specialsection{Введение}
\lipsum[1-5]
\specialsection{Постановка задачи}
\lipsum[1]

\begin{enumerate}
    \item \lipsum[5][1-2] % Первый пункт
    \item \lipsum[5][3-4] % Второй пункт
    \item \lipsum[6][1-2] % Третий пункт
    \begin{enumerate}
        \item \lipsum[7][1] % Вложенный нумерованный список второго уровня
        \item \lipsum[7][2] % Еще один пункт второго уровня
        \item \lipsum[7][3] % Последний пункт второго уровня
    \end{enumerate}
    \item \lipsum[6][3-4] % Четвертый пункт первого уровня
\end{enumerate}

\specialsection{Обзор}
\lipsum[1-5]
\specialsubsection{Обзор чего-то там}
\lipsum[1-5]
\section{Задача I}
\lipsum[1]
\subsection{Результат 1}
\lipsum[1-5]
\subsection{Результат 2}
\lipsum[1-5]
\section{Задача II}
\lipsum[1-5]
\specialsection{Заключение}
\lipsum[1]

\begin{itemize}
    \item \lipsum[9][1-2] % Маркированный пункт
    \begin{enumerate}
        \item \lipsum[10][1] % Нумерованный вложенный
        \item \lipsum[10][3] % Последний нумерованный
    \end{enumerate}
    \item \lipsum[9][3-4] % Последний маркированный
\end{itemize}

\end{document}
