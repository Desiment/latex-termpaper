\documentclass[14pt, gost]{termpaper}

\usepackage{lipsum}

\begin{document}
\maketitle

\pagebreak
\newfontfamily\myfont{CMU Sans Serif}
\begin{center}
    \hspace{0pt}
    \vfill
    {\Huge\myfont
        Текст ВКР или учебной практики пишется не ради зачета, а чтобы люди его прочитали, поняли как круто Вы все сделали, и могли продолжить с того места, где Вы остановились.
        \vspace{2em}

        Повторять эту страницу в тексте вашей работы \emph{нельзя}.}
    \vfill
    \hspace{0pt}
\end{center}
\pagebreak

\setcounter{tocdepth}{2}
\tableofcontents

\specialsection{Введение}
\lipsum[1-5]
\specialsection{Постановка задачи}
\lipsum[1]

\specialsection{Обзор}
\lipsum[1-5]
\specialsubsection{Обзор чего-то там}
\lipsum[1-5]
\section{Задача I}
\lipsum[1]
\subsection{Результат 1}
\lipsum[1-5]
\subsection{Результат 2}
\lipsum[1-5]
\section{Задача II}
\lipsum[1-5]
\specialsection{Заключение}
\lipsum[1]

\end{document}
